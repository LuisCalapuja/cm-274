\documentclass[10pt,a4paper]{article}
\usepackage[utf8]{inputenc}
\usepackage[spanish]{babel}
\usepackage{amsmath}
\usepackage{amsfonts}
\usepackage{amssymb}
\usepackage{graphicx}
\usepackage[left=2cm,right=2cm,top=2cm,bottom=2cm]{geometry}
\author{{*}Luis Alberto Calapuja Apaza \\lcalapujaa$@$uni.pe}
\title{Abstract}
\begin{document}
\maketitle
\begin{abstract}
A pesar de que fue en la década del 40 que las primeras computadoras modernas fueron
desarrolladas,  la  simulación  ya  existía  en  forma  embrionaria  aún  antes  de  que  la
computadora apareciera en escena. Así, por ejemplo, en la segunda mitad del siglo XIX,
se realizaban experiencias arrojando agujas al azar sobre una superficie reglada con el fin
de estimar el número 
$\pi$
.  En  1908  W.  S.  Gosset,  bajo  el  seudónimo  de  Student,  realizaba
un  muestreo  experimental  con  el  fin  de  descubrir  la  distribución  de  un  estimador  de  la
correlación en una distribución normal bivariada. En ese momento los números aleatorios
se  generaban  mediante  métodos  observacionales  (mecanismos  físicos)  tales  como  tirar
un dado, extraer una carta de un mazo o mediante una ruleta.
\end{abstract}

\end{document}