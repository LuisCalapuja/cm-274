\documentclass[10pt,a4paper]{article}
\usepackage[utf8]{inputenc}
\usepackage[spanish]{babel}
\usepackage{amsmath}
\usepackage{amsfonts}
\usepackage{amssymb}
\usepackage{graphicx}
\usepackage[left=2cm,right=2cm,top=2cm,bottom=2cm]{geometry}
\author{{*}Luis Alberto Calapuja Apaza \\lcalapujaa$@$uni.pe}
\title{Abstract}
\begin{document}
\maketitle
\begin{abstract}
Los números aleatorios son un elemento clave en múltiples procesos de la vida digital. Estos se utilizan no sólo en aplicaciones con gran componente de aleatoriedad como puede ser el juego online, sino que también tienen múltiples aplicaciones en el mundo de la ciberseguridad.En algunos sistemas criptográficos la seguridad depende de algo únicamente conocido por el personal autorizado pero impredecible para los atacantes, tal y como pasa con los sistemas de token electrónico
PROPOSITO los números pseudo-aleatorios son importantes en la práctica para simulaciones (por ejemplo, de sistemas físicos mediante el método de Montecarlo), y desempeñan un papel central en la criptografía. 
\end{abstract}

\end{document}